\documentclass[11pt]{artikel3}
\usepackage{fullpage, setspace, graphicx}
\usepackage[margin=1in]{geometry}
\usepackage{times}

\title{RIT Department of Computer Science\\MSc Project/Thesis Pre-Proposal:\\\emph{Proposed Project/Thesis Title}}
\author{FirstName LastName}
\date{\today}

\begin{document}
\maketitle

The sections shown below are adapted from the topic analysis forms provided in ``Writing the Doctoral Dissertation" (2nd edition) by Davis and Parker (pages 82-88). Your final document should be 1-2 pages including references. {\bf The final pre-proposal may present the items below in any format, but using prose (not bulleted lists).} 

\section{Problem Description}
Identify what problem you are addressing, both in terms of the research area, and the \emph{specific} problem that you will be working on:
\begin{itemize}
	\item For a thesis, a hypothesis (`thesis statement') that you will test in your research.
	\item For a project, identify the work required (e.g. implementation and/or experiment) that needs to be completed. If you are completing a project, make sure to speak with your advisor about the expected deliverable; one deliverable will be a written project report.
\end{itemize}

\section{Importance of Research}
Motivate your problem. 
\begin{itemize}
	\item What is the significance of your problem? 
	\item What applications or new opportunities will solving your problem provide?
\end{itemize}

\section{Related Work}

Demonstrate the connection between your chosen problem and how it is related to existing work. 
\begin{itemize}
	\item What are the key theoretical models (e.g. process-based, formal language/complexity models, probability-based) and algorithms have been applied toward this problem previously? 
	\item What limitation and/or opportunity do you plan to address in your project/thesis?
	\item
In the related research literature, how is success measured (e.g. metrics and/or coverage of problem aspects)?
\end{itemize}

\section{Methodology}

What theory, model, or algorithm do you plan to modify or develop to address your research problem?
\begin{itemize}
\item What methods/techniques will you use to address your problem? 
\begin{itemize}
\item For theory-based projects and theses, what are the key theorems to be developed and/or proven? What proof techniques will be used?
\item For projects and theses involving experiments, what metrics
will you use to measure success? Commonly these include some subset of time, space, and accuracy (recognition rate, precision, recall, etc.).
\end{itemize}

\item How you will you measure success? Almost always, this should include reference to the evaluation methods described in the related work.

\item {\bf How will you know when you are done?}
\end{itemize}

\section{Potential Outcomes}

\begin{itemize}
\item Given your chosen methods, what are the possible outcomes of the work? 
\item What is the contribution/significance of the result for each outcome? 
\end{itemize}

\bibliographystyle{plain}
\bibliography{preproposal}

**(Omitted) As an exercise, modify this document to include the references in the {\tt plain.bib} file.

\end{document}
