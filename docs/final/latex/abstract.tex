\begin{abstractpage}
Massive point cloud data sets are currently being created and studied in
academia, the private sector, and the military. Many previous attempts at
rendering point clouds have allowed the user to visualize the data in a
three-dimensional way but did not allow them to interact with the data and would
require all data to be in memory at runtime. Recently, a few systems have
emerged that deal with real-time rendering of massive point clouds with
on-the-fly level of detail modification that handles out-of-core processing but
these systems have their own limitations. With the size and scale of massive
point cloud data coming from LiDAR (Light Detection and Ranging) systems, being
able to visualize the data as well as interact and transform the data is needed.

Previous work in out-of-core rendering
\cite{3_wenzel2014out,4_goswami_zhang_pajarola_gobbetti_2010,5_richter_2010}
showed that using Octrees and k-d trees can increase the availability of data as
well as allow a user to visualize the information in a much more useful manner.
However, viewing the data isn't enough; applying work in context-aware selection
\cite{2_yu:hal-01178051} and surface creation \cite{1_VAST:VAST11:105-112} the
visualization system would greatly benefit in usability and functionality.

This paper explores a new data structure called an Icosatree, or icosahedral
tree, that can be used to partition a point cloud dataset in the same fashion as
an Octree is currently used. However, the Icosatree is made from triangular
prism sub-cells which are tangential to the ellipsoidal surface used by
Earth-based projected coordinate systems. In doing so, as new sub-cells are
added to the rendering system, a much more uniform visualization emerges. 

Along the same lines, this paper applies portions of the aforementioned
context-aware selection and surface creation algorithms to the resulting
visualization such that a user may triangulate, prune and/or export portions of
the point cloud dataset using an intuitive three-dimensional interface and
user-modifiable set of parameters. This allows the user to save items of
interest for later analysis.
\end{abstractpage}