\chapter{Design}
Previous work in the field of massive point cloud creation and processing has
moved toward data systems
\cite{6_krishnan_crosby_nandigam_phan_cowart_baru_arrowsmith_2011} such as
OpenTopography.org as well as research in leveraging Octree and k-d tree data
structures for taking massive point cloud data and partitioning it into
manageable pieces for the visualization system.
Separate work in selection and surface generation have increased the user
interaction and utility of these types of datasets. However, tree structures in
Cartesian or geodesic coordinate systems each have their own drawbacks and
limiting the user to visualization alone gives them a powerful tool that does
not allow the user to analyze the information further or export portions of the
data for use in more specialized software packages.

The first hurdle is to look at the state of data partitioning and access.
Currently, Octrees and k-d trees work in a Cartesian coordinate system. With
point cloud data that is relatively limited in size and scope, using an
arbitrary coordinate system can work well. Moving towards larger data sets and
rendering them on a geospatial projected surface causes the data to no longer
follow an axis-aligned format. This causes more issues with rendering such as
data structures that clip the projected surface and do not fill the partitioned
sections well. The first goal of this research is to modify the data structure
used to store the massive point cloud data from an Octree/k-d tree structure
where the XYZ Cartesian grid doesn't align well with a geospatial projected
surface (WGS84) into an icosahedral grid using previous work in spherical self
organizing grids \cite{7_wutakatsuka_2006} and traversal of triangle mesh
elements \cite{8_lee1998traversing} \cite{9_white_2000}.
Allowing the data to be split into surface aligned data structures should look
more pleasing to the end user as the level of detail is modified on-the-fly
compared to an Octree/k-d tree structure without the need of overdrawing or
forcing the user to wait until all data is available before rendering.

The second hurdle is to add functionality to the system instead of just being a
visualization system. Previous work in the area consisted of interacting with
unstructured point clouds \cite{2_yu:hal-01178051} by allowing the user the
ability to select more-dense regions of data using a screen-space masking
system. However, this implementation is limited by requiring the data be
unstructured, or containing a very heterogeneous density throughout.
Unfortunately, LiDAR data is rarely sparse; it is rarely pruned or processed at
all before being accessed by researchers. In another area, on-the-fly surface
creation has been applied to point cloud data in order to show physical
structures \cite{1_VAST:VAST11:105-112} in the input data; this shows off actual
objects in the virtual world without having extensive up-front processing of the
data. This surface modeling can be leveraged against the context-aware selection
algorithm in a uniformly dense point cloud to allow the user to select via
screen-space masking controls objects in the LiDAR point cloud data. From this,
the point data can be displayed separately for further study or exported for use
in other applications.

In order to evaluate these assumptions, a number of utility applications have
been developed as well as a simple visualization. The visualization has been
designed to load a point cloud and allow the end user to modify the rendering
settings, use a screen-selection lasso for object selection, and display and
export the selected points and resulting triangulation. The utilities developed
consist of command line applications that will access a binary file of point
data along with a CSV file of point attributes. The application then partitions
and exports the data as either an Octree or Icosatree for use with the
visualization application. A command line application was also created to
partition the binary input file into smaller portions for ease of use and
testing. Lastly, the visualization system was developed using a number of
previously developed graphics and computation libraries developed by the author
for previous work as well as a number of third party, open source, libraries.
