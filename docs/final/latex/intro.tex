\chapter{Introduction}

% % Obviously you need to delete these lines when you have written up your text

Massive point cloud datasets are becoming more prevalent as technology becomes
cheaper and storage and rendering power grows by leaps and bounds. This thesis
attempts to demonstrate a novel partitioning structure called an Icosatree as
well as a three dimensional triangular coordinate used for positioning values
inside the data structure. This icosahedron-based three dimensional data
partitioning structure was developed in order to address a few failings seen
with common data structures used for these types of datasets, such as the
Octree. The algorithm itself is explained further in Chapter 3.

First, the Octree being axis-aligned is poorly suited to the projected surface
the data is displayed in. This causes the cells in the tree to look asymmetrical
and as each subsequent layer of the tree is displayed the asymmetrical nature of
the visualization becomes more apparent. With the Icosatree, each cell is more
closely aligned with the projected geospatial surface and each cell appears as a
triangular area to the user. In the final visualization, the cells are added as
they are accessed from the data source (local file system or HTTP requests) and
as they are added to the rendering element it does indeed look much more
uniform.

Second, with the Icosatree being much closer to the projected geospatial surface
the hope is that less cells would be required in the output dataset and would
more efficiently fill our rendering data structure. This has been seen to be the
case, however, with limited resources on commodity graphics hardware a sweet
spot between performance and data utilization was more difficult to determine
automatically by the system and needs improvement. This will be covered in more
depth in the future work section at the end of this paper.

The Icosatree also uses a new triangular coordinate system in order to position
points into the Icosatree cells, or Icosatet. The Icosatet is a triangular prism
that is used as the basic sub-structure in the Icosatree. The initial Icosatree
is split into twenty Icosatets and each subsequent level in the tree splits each
Icosatet into eight smaller Icosatets. The triangular coordinates used allow the
points in each level of the Icosatree to be as uniformly distributed as
possible. This new coordinate system is explained in Chapter 4.

The resulting visualization was then used to test a hybrid user-definable
triangulation algorithm loosely based on the work found in two separate papers
[1,2]. The general idea behind this augmented algorithm is that the user selects
a region of the screen and any points within that area are selected. This set of
points will then be pruned using a combination of nearest neighbor density
values, eigenvector comparisons between the entire selection and nearest
neighbors, and altitude thresholds which are all definable by the user. Next,
the set can be limited further by removing any points hidden by others based on
a ray casting and occlusion test. Finally, the resulting set of points are fed
into a Delaunay triangulation algorithm in order to create a triangle mesh.